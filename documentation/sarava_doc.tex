\documentclass[a4paper]{article}
\usepackage{manfnt}
\usepackage{amsmath}
\usepackage{amssymb}
\usepackage{amsthm}
\usepackage{enumerate}
%\usepackage{enumitem}
\usepackage{tikz}
\usepackage{graphicx}
\usepackage[colorlinks]{hyperref}
\usepackage{textcomp}
\usepackage[retainorgcmds]{IEEEtrantools}
\theoremstyle{plain}
\newtheorem{theorem}{قضیه}
\newtheorem{lemma}{لم}
\usepackage{xepersian}
%\settextfont{B Nazanin}
%\setlatintextfont{Times New Roman}

\title{مستندات کتابخانه طراحی‌شده برای مسألهٔ مطرح‌شده در \lr{zconf}}
\author{سینا ممکن}
\date{95/06/18}

\begin{document}
\maketitle

\section*{تکنولوژی‌های بکاررفته}
برای نوشتن برنامهٔ جواب مسألهٔ مطرح‌شده توسط «سراواپارس»، من از زبان برنامه‌نویسی پایتون و محیط برنامه‌نویسی \lr{pycharm} استفاده کردم و تنها با استفاده از تعریف دو تابع ساده، هر دو کار اضافه کردنِ در لحظه به مجموعه لغات و دریافتِ خروجی متناظر با ورودی را پیاده‌سازی کردم.

جهت نوشتن \lr{unit test}ها نیز از کتابخانهٔ ساده‌ٔ \lr{doctest} که در پایتون وجود دارد استفاده کرده‌ایم.

در مورد الگوریتم به کار رفته در برنامه نیز من به اینصورت عمل کرده‌ام که به ازای هر کلمه که به مجموعهٔ لغات انگلیسی ما اضافه می‌شود، در همان لحظه عدد متناظر با آن کلمه را بهمراه خود کلمه در یک دیکشنری پایتون اضافه می‌کنیم. با توجه به اینکه در دیکشنریِ ما به ازای یک عدد می‌تواند چندین کلمه وجود داشته باشد، بنابراین هنگام اضافه کردن به این دیکشنری، کلید، عدد متناظر با کلمه و مقدار، لیستی از تمام کلمات متناظر با آن عدد خواهند بود.
\\
این الگوریتم ساده بوده و در نتیجه پیاده‌سازی آن سریع خواهد بود؛ بنابراین ما از آن به جای پردازش عدد ورودی به صورت درختی استفاده کرده‌ایم. ضمن اینکه خود دیکشنری دارای مرتبهٔ زمانی جستجو $O(1)$ می‌باشد و در نتیجه هنگام دادن ورودی نیز، برنامه از بالاترین کارایی زمانی برخوردار است.

نهایتاً وقتی کاربر عددی را به عنوان ورودی به تابع \lr{give\_outputs\_of} می‌دهد، تنها آن عدد در دیکشنری پیدا شده و لیست تمام کلمات متناظر با آن عدد به عنوان خروجی داده خواهد شد.

با توجه به استفاده از پایتون، طبیعتاً از \lr{code style} خود پایتون یعنی \lr{PEP8} استفاده شده است.

برای \lr{unit test}ها نیز ۳ تا تست تعریف شده‌اند (که یکی از آنها همان مثال مطرح شده در خود سوال است). برای اینکه نتیجهٔ \lr{unit test}ها را ببینید تنها کافی است دستور
\begin{latin}
	\begin{verbatim}
		python3 -i sarava.py
	\end{verbatim}
\end{latin}
را در ترمینال، درون پوشه‌ای که فایل \lr{sarava.py} در آن قرار دارد، اجرا کنید.
\end{document}